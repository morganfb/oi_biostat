%!TEX root=../../main.tex

%_______________
\section{Exercises}



%_______________
\subsection{Defining probability}

% oi_biostat, edited

\eoce{\qt{True or false\label{tf_prob_definitions}} Determine if the statements 
below are true or false, and explain your reasoning.
\begin{parts}
\item Assume that a couple has an equal chance of having a boy or a girl. If a couple's previous three children have all been boys, then the chance that their next child is a boy is somewhat less than 50\%.
\item Drawing a face card (jack, queen, or king) and drawing a red card from a 
full deck of playing cards are mutually exclusive events.
\item Drawing a face card and drawing an ace from a full deck of playing cards 
are mutually exclusive events.
\end{parts}
}{}

% 6

\eoce{\qt{Dice rolls\label{dice_rolls}} If you roll a pair of fair dice, what is 
	the probability of
	\begin{parts}
		\item getting a sum of 1?
		\item getting a sum of 5?
		\item getting a sum of 12?
	\end{parts}
}{}

%oi_biostat

\eoce{\qt{Colorblindness\label{colorblindness}} Red-green colorblindness is a commonly inherited form of colorblindness; the gene involved is transmitted on the X chromosome in a recessive manner. If a male inherits an affected X chromosome, he is necessarily colorblind (genotype $X^{-}Y$). However, a female can only be colorblind if she inherits two defective copies (genotype $X^{-}X^{-}$); heterozygous females are not colorblind. Suppose that a couple consists of a genotype $X^{+}Y$ male and a genotype $X^{+}X^{-}$ female.
\begin{parts}
\item What is the probability of the couple producing a colorblind male?
\item True or false: Among the couple's offspring, colorblindness and female sex are mutually exclusive events.
\end{parts}
}{}

%oi_biostat

\eoce{\qt{Diabetes and hypertension\label{diabetes_hypertension}} Diabetes and hypertension are two of the most common diseases in Western, industrialized nations. In the United States, approximately 9\% of the population have diabetes, while about 30\% of adults have high blood pressure. The two diseases frequently occur together: an estimated 6\% of the population have both diabetes and hypertension.
\begin{parts}
\item Are having diabetes and having hypertension disjoint?
\item Draw a Venn diagram summarizing the variables and their associated probabilities.
\item Let $A$ represent the event of having diabetes, and $B$ the event of having hypertension. Calculate $P(A \text{ or } B)$. 
\item What percent of Americans have neither hypertension nor diabetes?
\item Is the event of someone being hypertensive independent of the event that someone has diabetes?
\end{parts}
}{}

% 3 million who have both in 1992, out of 256 million ~ 1.1%
% 2012: approx 9% have diabetes http://www.diabetes.org/diabetes-basics/statistics/
% approx 1/3 Americans have hypertension: 30\% http://www.diabetes.org/are-you-at-risk/lower-your-risk/bloodpressure.html?referrer=https://www.google.com/

% 8

\eoce{\qt{Poverty and language\label{poverty_language}} The American Community 
Survey is an ongoing survey that provides data every year to give communities the 
current information they need to plan investments and services. The 2010 American 
Community Survey estimates that 14.6\% of Americans live below the poverty line, 
20.7\% speak a language other than English (foreign language) at home, and 4.2\% 
fall into both categories. \footfullcite{poorLang}
\begin{parts}
\item Are living below the poverty line and speaking a foreign language at home 
disjoint?
\item Draw a Venn diagram summarizing the variables and their associated 
probabilities.
\item What percent of Americans live below the poverty line and only speak 
English at home?
\item What percent of Americans live below the poverty line or speak a foreign 
language at home?
\item What percent of Americans live above the poverty line and only speak 
English at home? 
\item Is the event that someone lives below the poverty line independent of the 
event that the person speaks a foreign language at home?
\end{parts}
}{}

% oi, edited

\eoce{\qt{Educational attainment by gender\label{edu_attain_couples}} The table 
below shows the distribution of education level attained by US residents by 
gender based on data collected during the 2010 American Community Survey.
\footfullcite{eduSex}
\begin{center}
\begin{tabular}{l p{7cm} c c }
&                                       & \multicolumn{2}{c}{\textit{Gender}} \\
\cline{3-4}
&                                                   & Male  & Female \\
\cline{2-4}
& Less than 9th grade                               & 0.07  & 0.13 \\
& 9th to 12th grade, no diploma                     & 0.10  & 0.09 \\
\textit{Highest}    & HS graduate (or equivalent)   & 0.30  & 0.20 \\
\textit{education}  & Some college, no degree       & 0.22  & 0.24 \\ 
\textit{attained}   & Associate's degree            & 0.06  & 0.08 \\
& Bachelor's degree                                 & 0.16  & 0.17 \\
& Graduate or professional degree                   & 0.09  & 0.09 \\
\cline{2-4} 
& Total                                             & 1.00  & 1.00
\end{tabular}
\end{center}
\begin{parts}
\item What is the probability that a randomly chosen individual is a high school graduate? Assume that there is an equal proportion of males and females in the population.
\item Define Event $A$ as having a graduate or professional degree. Calculate the probability of the complement, $A^c$.
\item What is the probability that a randomly chosen man has at least a 
Bachelor's degree?
\item What is the probability that a randomly chosen woman has at least a 
Bachelor's degree?
\item What is the probability that a man and a woman getting married both have at 
least a Bachelor's degree? Note any assumptions made -- are they reasonable?
\end{parts}
}{}

% oi, edited

\eoce{\qt{School absences\label{school_absences}} Data collected at elementary 
schools in DeKalb County, GA suggest that each year roughly 25\% of students miss 
exactly one day of school, 15\% miss 2 days, and 28\% miss 3 or more days due to 
sickness. \footfullcite{Mizan:2011}
\begin{parts}
\item What is the probability that a student chosen at random doesn't miss any 
days of school due to sickness this year?
\item What is the probability that a student chosen at random misses no more than 
one day?
\item What is the probability that a student chosen at random misses at least one 
day?
\item If a parent has two kids at a DeKalb County elementary school, what is the 
probability that neither kid will miss any school? Note any assumptions made and evaluate how reasonable they are.
\item If a parent has two kids at a DeKalb County elementary school, what is the 
probability that both kids will miss some school, i.e. at least one day? Note any assumptions made and evaluate how reasonable they are.
\end{parts}
}{}

% oi_biostat

\eoce{\qt{Urgent care visits\label{urgent_care}} Urgent care centers are open beyond typical office hours and provide a broader range of services than that of many primary care offices. A study conducted to collected information about urgent care centers in the United States reported that in one week, 15.8\% of centers saw 0-149 patients, 33.7\% saw 150-299 patients, 28.8\% saw 300-449 patients, and 21.7\% saw 450 or more patients. Assume that the data can be treated as a probability distribution of patient visits for any given week. 
\begin{parts}	
\item What is the probability that three random urgent care centers in a county all see between 300-449 patients in a week? Note any assumptions made. Are the assumptions reasonable?
\item What is the probability that ten random urgent care centers throughout a state all see 450 or more patients in a week? Note any assumptions made. Are the assumptions reasonable?
\item With the information provided, is it possible to compute the probability that one urgent care center sees between 150-299 patients in one week and 300-449 patients in the next week? Explain why or why not.
\end{parts}	
}{}

% data from urgent_care_visits

% oi, edited

\eoce{\qt{Health coverage, frequencies\label{health_coverage_freqs}} The 
Behavioral Risk Factor Surveillance System (BRFSS) is an annual telephone survey 
designed to identify risk factors in the adult population and report emerging 
health trends. The following table summarizes two variables for the respondents: 
health status and health coverage, which describes whether each respondent had 
health insurance. \footfullcite{data:BRFSS2010}
\begin{center}
\begin{tabular}{rrrrrrrr}
                    &       & \multicolumn{5}{c}{\textit{Health Status}} &  \\ 
\cline{3-7}
                    &       & Excellent & Very good & Good  & Fair  & Poor  & Total\\ 
\cline{2-8}
\textit{Health}     & No    & 459       & 727       & 854   & 385   & 99    & 2,524 \\ 
\textit{Coverage}   & Yes   & 4,198     & 6,245     & 4,821 & 1,634 & 578   & 17,476 \\ 
\cline{2-8}
                    & Total & 4,657     & 6,972     & 5,675 & 2,019 & 677   & 20,000
\end{tabular}
\end{center}
\begin{parts}
\item If one individual is drawn at random, what is the probability that the 
respondent has excellent health and doesn't have health coverage?
\item If one individual is drawn at random, what is the probability that the 
respondent has excellent health or doesn't have health coverage?
\end{parts}
}{}



%_______________
\subsection{Conditional probability}

% oi, edited

\eoce{\qt{Global warming\label{global_warming}} A 2010 Pew Research poll asked 
1,306 Americans ``From what you've read and heard, is there solid evidence that 
the average temperature on earth has been getting warmer over the past few 
decades, or not?". The table below shows the distribution of responses by party 
and ideology, where the counts have been replaced with relative frequencies.
\footfullcite{globalWarming}
\begin{center}
\begin{tabular}{ll  ccc c} 
                    &                           & \multicolumn{3}{c}{\textit{Response}} \\
\cline{3-5}
                    &                           & Earth is  & Not       & Don't Know    &   \\
                    &                           & warming   & warming   & Refuse        & Total\\
\cline{2-6}
                    & Conservative Republican   & 0.11      & 0.20      & 0.02      & 0.33  \\
\textit{Party and}  & Mod/Lib Republican        & 0.06      & 0.06      & 0.01      & 0.13 \\
\textit{Ideology}   & Mod/Cons Democrat         & 0.25      & 0.07      & 0.02      & 0.34 \\
                    & Liberal Democrat          & 0.18      & 0.01      & 0.01      & 0.20\\
\cline{2-6}
                    &Total                      & 0.60      & 0.34      & 0.06      & 1.00
\end{tabular}
\end{center}
\begin{parts}
\item What is the probability that a randomly chosen respondent believes the 
earth is warming or is a liberal Democrat?
\item What is the probability that a randomly chosen respondent believes the 
earth is warming given that they are a liberal Democrat?
\item What is the probability that a randomly chosen respondent believes the 
earth is warming given that they are a conservative Republican?
\item Does it appear that whether or not a respondent believes the earth is 
warming is independent of their party and ideology? Explain your reasoning.
\item What is the probability that a randomly chosen respondent is a 
moderate/liberal Republican given that they does not believe that the earth is 
warming? 
\end{parts}
}{}

% 18

\eoce{\qt{Health coverage, relative frequencies\label{health_coverage_rel_freqs}} 
The Behavioral Risk Factor Surveillance System (BRFSS) is an annual telephone 
survey designed to identify risk factors in the adult population and report 
emerging health trends. The following table displays the distribution of health 
status of respondents to this survey (excellent, very good, good, fair, poor) 
conditional on whether or not they have health insurance.
\begin{center}
\begin{tabular}{rrrrrrrr}
& &  \multicolumn{5}{c}{\textit{Health Status}} &  \\ 
\cline{3-7}
                    &       & Excellent & Very good & Good      & Fair      & Poor      & Total \\ 
\cline{2-8}
\textit{Health}     & No    & 0.0230    & 0.0364    & 0.0427    & 0.0192    & 0.0050    & 0.1262 \\ 
\textit{Coverage}   & Yes   & 0.2099    & 0.3123    & 0.2410    & 0.0817    & 0.0289    & 0.8738 \\ 
\cline{2-8}
                    & Total & 0.2329    & 0.3486    & 0.2838    & 0.1009    & 0.0338    & 1.0000
\end{tabular}
\end{center}
\begin{parts}
\item Are being in excellent health and having health coverage mutually 
exclusive?
\item What is the probability that a randomly chosen individual has excellent 
health?
\item What is the probability that a randomly chosen individual has excellent 
health given that he has health coverage?
\item What is the probability that a randomly chosen individual has excellent 
health given that he doesn't have health coverage?
\item Do having excellent health and having health coverage appear to be 
independent?
\end{parts}
}{}


%JV: Stopped here, 14July

% oi_biostat

\eoce{\qt{Seat belts\label{seat_belts}}
Seat belt use is the most effective way to save lives and reduce injuries in motor vehicle crashes. In a 2014 survey, respondents were asked, "How often do you use seat belts when you drive or ride in a car?". The following table shows the distribution of seat belt usage by sex.
\begin{center}
	\begin{tabular}{rrrrrrrr}
		& &  \multicolumn{5}{c}{\textit{Seat Belt Usage}} &  \\ 
		\cline{3-7}
		&       & Always & Nearly always & Sometimes    & Seldom     & Never  & Total \\ 
		\cline{2-8}
		\multirow{2}{*}{\textit{Sex}}    & Male    & 146,018   & 19,492    & 7,614   &  3,145  & 4,719 & 180,988 \\ 
					  & Female   & 229,246    & 16,695    & 5,549    & 1,815  & 2,675 &  256,25 \\ 
		\cline{2-8}
		& Total & 375,264    & 36,457    & 13,163    & 4,960   & 7,394  &  437,238
	\end{tabular}
\end{center}
\begin{parts}
\item Calculate the marginal probability that a randomly chosen individual always wears seatbelts.
\item What is the probability that a randomly chosen female always wears seatbelts?
\item What is the conditional probability of a randomly chosen individual always wearing seatbelts, given that they are female? 
\item What is the conditional probability of a randomly chosen individual always wearing seatbelts, given that they are male?
\item Calculate the probability that an individual who never wears seatbelts is male.
\item Does gender seem independent of seat belt usage?
\end{parts}
}{}

% 2014 brfss data

% 20

\eoce{\qt{Assortative mating\label{assortative_mating}} Assortative mating is a 
nonrandom mating pattern where individuals with similar genotypes and/or 
phenotypes mate with one another more frequently than what would be expected 
under a random mating pattern. Researchers studying this topic collected data on 
eye colors of 204 Scandinavian men and their female partners. The table below 
summarizes the results. For simplicity, we only include heterosexual 
relationships in this exercise. \footfullcite{Laeng:2007}
\begin{center}
\begin{tabular}{ll  ccc c} 
                                        &           & \multicolumn{3}{c}{\textit{Partner (female)}} \\
\cline{3-5}
                                        &           & Blue  & Brown     & Green     & Total \\
\cline{2-6}
                                        & Blue      & 78    & 23        & 13        & 114 \\
\multirow{2}{*}{\textit{Self (male)}}   & Brown     & 19    & 23        & 12        & 54 \\
                                        & Green     & 11    & 9         & 16        & 36 \\
\cline{2-6}  
                                        & Total     & 108   & 55        & 41        & 204
\end{tabular}
\end{center}
\begin{parts}
\item What is the probability that a randomly chosen male respondent or his 
partner has blue eyes?
\item What is the probability that a randomly chosen male respondent with blue 
eyes has a partner with blue eyes? 
\item What is the probability that a randomly chosen male respondent with brown 
eyes has a partner with blue eyes? What about the probability of a randomly 
chosen male respondent with green eyes having a partner with blue eyes?
\item Does it appear that the eye colors of male respondents and their partners 
are independent? Explain your reasoning.
\end{parts}
}{}

% 22

\eoce{\qt{Predisposition for thrombosis\label{tree_thrombosis}} A genetic test is 
used to determine if people have a predisposition for \textit{thrombosis}, which 
is the formation of a blood clot inside a blood vessel that obstructs the flow of 
blood through the circulatory system. It is believed that 3\% of people actually 
have this predisposition. The genetic test is 99\% accurate if a person actually 
has the predisposition, meaning that the probability of a positive test result 
when a person actually has the predisposition is 0.99. The test is 98\% accurate 
if a person does not have the predisposition. What is the probability that a 
randomly selected person who tests positive for the predisposition by the test 
actually has the predisposition?
}{}

% 23

\eoce{\qt{HIV in Swaziland\label{tree_hiv_swaziland}} Swaziland has the highest 
HIV prevalence in the world: 25.9\% of this country's population is infected with 
HIV.\footfullcite{ciaFactBookHIV:2012} The ELISA test is one of the first and 
most accurate tests for HIV. For those who carry HIV, the ELISA test is 99.7\% 
accurate. For those who do not carry HIV, the test is 92.6\% accurate. If an 
individual from Swaziland has tested positive, what is the probability that he 
carries HIV?
}{}


%oi_biostat

\eoce{\qt{Views on evolution\label{views_evolution}} A 2013 analysis conducted by the Pew Research Center found that 60\% of survey respondents agree with the statement "humans and other living things have evolved over time" while 33\% say that "humans and other living things have existed in their present form since the beginning of time" (7\% responded "don't know"). They also found that there are differences among partisan groups in beliefs about evolution. While roughly two-thirds of Democrats (67\%) and independents (65\%) say that humans and other living things have evolved over time, 48\% of Republicans reject the idea of evolution. Suppose that 45\% of respondents identified as Democrats, 40\% identified as Republicans, and 15\% identified as political independents. The survey was conducted among a national sample of 1,983 adults.
\begin{parts}
\item Suppose that a person is randomly selected from the population and found to identify as a Democrat. What is the probability that this person does not support the idea of evolution?
\item Suppose that a political independent is randomly selected from the population. What is the probability that this person supports the idea of evolution?
\item Suppose that a person is randomly selected from the population and found to identify as a Republican. What is the probability that this person supports the idea of evolution?
\end{parts}
}{}

%http://www.pewforum.org/2013/12/30/publics-views-on-human-evolution/

%oi_biostat

\eoce{\qt{Cystic fibrosis testing\label{cf_testing}} The prevalence of cystic fibrosis in the United States is approximately 1 in 3,500 births. Various screening strategies for CF exist. One strategy uses dried blood samples to check the levels of immunoreactive trypsogen (IRT); IRT levels are commonly elevated in newborns with CF. The sensitivity of the IRT screen is 87\% and the specificity is 99\%. 
\begin{parts}
\item In a hypothetical population of 100,000, how many individuals would be expected to test positive? Of those who test positive, how many would be true positives? Calculate the PPV of IRT.
\item In order to account for lab error or physiological fluctuations in IRT levels, infants who tested positive on the initial IRT screen are asked to return for another IRT screen at a later time, usually two weeks after the first test. This is referred to as an IRT/IRT screening strategy. Calculate the PPV of IRT/IRT.
\end{parts}
}{}

%from 2016 final exam

% 25

\eoce{\qt{It's never lupus\label{tree_lupus}} Lupus is a medical phenomenon where 
antibodies that are supposed to attack foreign cells to prevent infections 
instead see plasma proteins as foreign bodies, leading to a high risk of blood 
clotting. It is believed that 2\% of the population suffer from this disease. The 
test is 98\% accurate if a person actually has the disease. The test is 74\% 
accurate if a person does not have the disease. There is a line from the Fox 
television show \emph{House} that is often used after a patient tests positive 
for lupus: ``It's never lupus." Do you think there is truth to this statement? 
Use appropriate probabilities to support your answer.
}{}

% 26

\eoce{\qt{Twins\label{tree_twins}} About 30\% of human twins are identical, and 
the rest are fraternal. Identical twins are necessarily the same sex -- half are 
males and the other half are females. One-quarter of fraternal twins are both 
male, one-quarter both female, and one-half are mixes: one male, one female. You 
have just become a parent of twins and are told they are both girls. Given this 
information, what is the probability that they are identical?
}{}

%_______________
\subsection{Random variables}

%oi_biostat

\eoce{\qt{Gull clutch size\label{gull_eggs}}
Large black-tailed gulls usually lay one to three eggs, and rarely have a four egg clutch. It is thought that clutch sizes are effectively limited by how effectively parents can incubate their eggs. Suppose that on average, gulls have a 25\% of laying 1 egg, 40\% of laying 2 eggs, 30\% chance of laying 3 eggs, and 5\% chance of laying 4 eggs.
\begin{parts}
\item Calculate the expected number of eggs laid by a random sample of 100 gulls. 
\item Calculate the standard deviation of the number of eggs laid by a random sample of 100 gulls.
\end{parts}	
}{}

% scenario from incubation_capacity

% 35, oi_edited

\eoce{\qt{Hearts win\label{hearts}} In a card game, the player starts with a well-
shuffled full deck and draw 3 cards without replacement. If the player draw 3 hearts, 
they win \$50. If they draw 3 black cards, they win \$25. For any other draws, nothing is won.

\begin{parts}
\item Create a probability model for the amount of money that can be won playing this game, and find the expected winnings. Also, compute the standard deviation of this distribution.
\item If the game costs \$5 to play, what would be the expected value and 
standard deviation of the net profit (or loss)? 
\item If the game costs \$5 to play, is it advantageous to play this game? Explain.
\end{parts}
}{}

% 38

\eoce{\qt{Baggage fees\label{baggage_fees}} An airline charges the following 
baggage fees: \$25 for the first bag and \$35 for the second. Suppose 54\% of 
passengers have no checked luggage, 34\% have one piece of checked luggage and 
12\% have two pieces. Suppose that a negligible portion of people check more than 
two bags.
\begin{parts}
\item Build a probability model, compute the average revenue per passenger, and 
compute the corresponding standard deviation.
\item About how much revenue should the airline expect for a flight of 120 
passengers? With what standard deviation? Note any assumptions made and whether they are justified.
\end{parts}
}{}

% 42

\eoce{\qt{Scooping ice cream\label{scoop_ice_cream}} Ice cream usually comes in 1.
5 quart boxes (48 fluid ounces), and ice cream scoops hold about 2 ounces. 
However, there is some variability in the amount of ice cream in a box as well as 
the amount of ice cream scooped out. We represent the amount of ice cream in the 
box as $X$ and the amount scooped out as $Y$. Suppose these random variables have 
the following means, standard deviations, and variances:
\begin{center}
\begin{tabular}{l ccc}
\hline
    & mean & SD & variance \\
\hline
$X$ & 48       & 1      & 1     \\
$Y$ & 2    & 0.25   & 0.0625    \\
\hline
\end{tabular}
\end{center}
\begin{parts}
\item An entire box of ice cream, plus 3 scoops from a second box is served at a 
party. How much ice cream do you expect to have been served at this party? What 
is the standard deviation of the amount of ice cream served?
\item How much ice cream would you expect to be left in the box after scooping 
out one scoop of ice cream? That is, find the expected value of $X-Y$. What is 
the standard deviation of the amount left in the box?
\item Using the context of this exercise, explain why we add variances when we 
subtract one random variable from another.
\end{parts}
}{}


%_______________
\subsection{Distributions for pairs of random variables}

%_______________
\subsection{Continuous distributions}

% 44

\eoce{\qt{Income and gender\label{income_gender}} The relative frequency table 
below displays the distribution of annual total personal income (in 2009 
inflation-adjusted dollars) for a representative sample of 96,420,486 Americans. 
These data come from the American Community Survey for 2005-2009. This sample is 
comprised of 59\% males and 41\% females. \footfullcite{acsIncome2005-2009} \\

\noindent\begin{minipage}[c]{0.60\textwidth}
\begin{parts}
\item Describe the distribution of total personal income.
\item What is the probability that a randomly chosen US resident makes less than 
\$50,000 per year?
\item What is the probability that a randomly chosen US resident makes less than 
\$50,000 per year and is female? Note any assumptions you make.
\item The same data source indicates that 71.8\% of females make less than 
\$50,000 per year. Use this value to determine whether or not the assumption you 
made in part (c) is valid.
\end{parts} 
\end{minipage}
\begin{minipage}[c]{0.4\textwidth}
{\small
\begin{center}
\begin{tabular}{lr}
  \hline
\textit{Income}         & \textit{Total} \\
  \hline
\$1 to \$9,999 or loss  & 2.2\% \\
\$10,000 to \$14,999    & 4.7\% \\
\$15,000 to \$24,999    & 15.8\% \\
\$25,000 to \$34,999    & 18.3\% \\
\$35,000 to \$49,999    & 21.2\% \\
\$50,000 to \$64,999    & 13.9\% \\
\$65,000 to \$74,999    & 5.8\% \\
\$75,000 to \$99,999    & 8.4\% \\
\$100,000 or more       & 9.7\% \\
   \hline
\end{tabular}
\end{center}
}
\end{minipage}
}{}