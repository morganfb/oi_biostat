


%_______________
\section{Exercises}

\noindent
\textit{JV: Edited on 03Sept2015. Deleted exercises that seemed too easy or covered eliminated topics (ex. mapping data). Notes about replacing exercises visible in pdf, original exercises to be modified in comments. Figure/graphic pathways edited to oi\_biostat. Section names updated.}

%_______________
\subsection{Case study: preventing peanut allergies}

% 1

\eoce{\qt{Migraine and acupuncture\label{migraine_and_acupuncture_intro}} A migraine 
is a particularly painful type of headache, which patients sometimes wish to 
treat with acupuncture. To determine whether acupuncture relieves migraine 
pain, researchers conducted a randomized controlled study where 89 females 
diagnosed with migraine headaches were randomly assigned to one of two groups: 
treatment or control. 43 patients in the treatment group received acupuncture 
that is specifically designed to treat migraines. 46 patients in the control 
group received placebo acupuncture (needle insertion at non-acupoint 
locations). 24 hours after patients received acupuncture, they were asked 
if they were pain free. Results are summarized in the contingency table below. 
\footfullcite{Allais:2011}

\noindent\begin{minipage}[l]{0.4\textwidth}
\begin{tabular}{ll  cc c} 
			                         &           & \multicolumn{2}{c}{\textit{Pain free}} \\
\cline{3-4}
			                         &			 & Yes 	    & No 	                      & Total	\\
\cline{2-5}
							         & Treatment & 10	 	& 33		                  & 43 	\\
\raisebox{1.5ex}[0pt]{\emph{Group}}  & Control	 & 2	 	& 44 	 	                  & 46 \\
\cline{2-5}
							         &Total		 & 12		& 77		                  & 89
\end{tabular}
\end{minipage}
\begin{minipage}[c]{0.05\textwidth}
\end{minipage}
\begin{minipage}[c]{0.27\textwidth}
\begin{center}
\includegraphics[width = 0.75\textwidth]{ch_intro_to_data_oi_biostat/figures/eoce/migraine_and_acupuncture_intro/earacupuncture.pdf}
\end{center}
\end{minipage}
\begin{minipage}[c]{0.25\textwidth}
{\footnotesize Figure from the original paper displaying the appropriate area 
(M) versus the inappropriate area (S) used in the treatment of migraine attacks.}
\end{minipage}
\begin{parts}
\item What percent of patients in the treatment group were pain free 24 hours 
after receiving acupuncture? What percent in the control group?
\item At first glance, does acupuncture appear to be an effective treatment for 
migraines? Explain your reasoning.
\item Do the data provide convincing evidence that there is a real pain reduction 
for those patients in the treatment group? Or do you think that the observed 
difference might just be due to chance?
\end{parts}
}{}

% 2

\eoce{\qt{Sinusitis and antibiotics\label{sinusitis_and_antibiotics_intro}} 
Researchers studying the effect of antibiotic treatment for acute sinusitis 
compared to symptomatic treatments randomly assigned 166 adults diagnosed 
with acute sinusitis to one of two groups: treatment or control. Study 
participants received either a 10-day course of amoxicillin (an antibiotic) 
or a placebo similar in appearance and taste. The placebo consisted of 
symptomatic treatments such as acetaminophen, nasal decongestants, etc. At the 
end of the 10-day period patients were asked if they experienced significant 
improvement in symptoms. The distribution of responses are summarized below. 
\footfullcite{Garbutt:2012}
\begin{center}
\begin{tabular}{ll  cc c} 
                                    &			& \multicolumn{2}{c}{\textit{Self-reported significant}} \\
			                        &			& \multicolumn{2}{c}{\textit{improvement in symptoms}} \\
\cline{3-4}
			                        &			& Yes 	& No 	& Total	\\
\cline{2-5}
							        & Treatment & 66	& 19	& 85 \\
\raisebox{1.5ex}[0pt]{\emph{Group}}	& Control	& 65	& 16 	& 81 \\
\cline{2-5}
							        & Total		& 131	& 35	& 166
\end{tabular}
\end{center}
\begin{parts}
\item What percent of patients in the treatment group experienced a significant 
improvement in symptoms? What percent in the control group?
\item Based on your findings in part (a), which treatment appears to be more 
effective for sinusitis?
\item Do the data provide convincing evidence that there is a difference in the 
improvement rates of sinusitis symptoms? Or do you think that the observed 
difference might just be due to chance?
\end{parts}
}{}



%_______________
\subsection{Data basics}

% 3

\eoce{\qt{Air pollution and birth outcomes, study components\label{study_components_airpoll}} 
Researchers collected data to examine the relationship between air pollutants 
and preterm births in Southern California. During the study air pollution levels 
were measured by air quality monitoring stations. Specifically, levels of carbon 
monoxide were recorded in parts per million, nitrogen dioxide and ozone in parts 
per hundred million, and coarse particulate matter (PM$_{10}$) in $\mu g/m^3$. 
Length of gestation data were collected on 143,196 births between the years 1989 
and 1993, and air pollution exposure during gestation was calculated for each 
birth. The analysis suggested that increased ambient PM$_{10}$ and, to a lesser 
degree, CO concentrations may be associated with the occurrence of preterm births. 
\footfullcite{Ritz+Yu+Chapa+Fruin:2000} Identify \begin{inparaenum}[(a)]
	\item the cases,
	\item the variables and their types, and
	\item the main research question
\end{inparaenum} 
	in this study.
}{}

% 4

\eoce{\qt{Buteyko method, study components\label{study_components_buteyko}} 
The Buteyko method is a shallow breathing technique developed by Konstantin 
Buteyko, a Russian doctor, in 1952. Anecdotal evidence suggests that the Buteyko 
method can reduce asthma symptoms and improve quality of life. In a scientific 
study to determine the effectiveness of this method, researchers recruited 600 
asthma patients aged 18-69 who relied on medication for asthma treatment. These 
patients were split into two research groups: one practiced the Buteyko method 
and the other did not. Patients were scored on quality of life, activity, 
asthma symptoms, and medication reduction on a scale from 0 to 10. On average, 
the participants in the Buteyko group experienced a significant reduction in 
asthma symptoms and an improvement in quality of life. \footfullcite{McDowan:2003} 
Identify \begin{inparaenum}[(a)]
\item the cases,
\item the variables and their types, and
\item the main research question 
\end{inparaenum}
in this study.
}{}

\noindent
\textit{JV: Placeholder for two biology-related questions asking for cases, variables, and research question.}

\begin{comment}
% 5

\eoce{\qt{Cheaters, study components\label{study_components_cheaters}} 
Researchers studying the relationship between honesty, age and 
self-control conducted an experiment on 160 children between the ages of 5 and 
15. Participants reported their age, sex, and whether they were an only child 
or not. The researchers asked each child to toss a fair coin in private and 
to record the outcome (white or black) on a paper sheet, and said they 
would only reward children who report white. Half the students were explicitly 
told not to cheat and the others were not given any explicit instructions. In 
the no instruction group probability of cheating was found to be uniform 
across groups based on child's characteristics. In the group that was 
explicitly told to not cheat, girls were less likely to cheat, and while rate 
of cheating didn't vary by age for boys, it decreased with age for girls. 
\footfullcite{Bucciol:2011} Identify \begin{inparaenum}[(a)]
\item the cases,
\item the variables and their types, and
\item the main research question 
\end{inparaenum}
in this study.
}{}

% 6

\eoce{\qt{Stealers, study components\label{study_components_stealers}} 
In a study of the relationship between socio-economic class and unethical 
behavior, 129 University of California undergraduates at Berkeley were asked 
to identify themselves as having low or high social-class by comparing 
themselves to others with the most (least) money, most (least) education, and 
most (least) respected jobs. They were also presented  with a jar of 
individually wrapped candies and informed that the candies were for children
in a nearby laboratory, but that they could take some if they wanted. After 
completing some unrelated tasks, participants reported the number of candies 
they had taken. It was found that those who were identified as upper-class 
took more candy than others. \footfullcite{Piff:2012} Identify
\begin{inparaenum}[(a)]
\item the cases,
\item the variables and their types, and
\item the main research question 
\end{inparaenum}
in this study.
}{}

\end{comment}

% 7

\eoce{\qt{Fisher's irises\label{fisher_irises}} Sir Ronald Aylmer Fisher was an 
English statistician, evolutionary biologist, and geneticist who worked on a 
dataset that contained sepal length and width, and petal length and width from 
three species of iris flowers (\textit{setosa}, \textit{versicolor} and 
\textit{virginica}). There were 50 flowers from each species in the data set. 
\footfullcite{Fisher:1936} \\
\noindent\begin{minipage}[c]{0.48\textwidth}
\begin{parts}
\item How many cases were included in the data?
\item How many numerical variables are included in the data? Indicate what 
they are, and if they are continuous or discrete.
\item How many categorical variables are included in the data, and what are 
they? List the corresponding levels (categories).
\end{parts}
\vfill \ 
\end{minipage}
\begin{minipage}[c]{0.01\textwidth}
\ 
\end{minipage}
\begin{minipage}[c]{0.2\textwidth}
\begin{center}
\includegraphics[width = \textwidth]{ch_intro_to_data_oi_biostat/figures/eoce/fisher_irises/irisversicolor.jpg}
\end{center}
\end{minipage}
\begin{minipage}[c]{0.01\textwidth}
\ 
\end{minipage}
\begin{minipage}[c]{0.23\textwidth}
{\raggedright\footnotesize Photo by Ryan Claussen 
(\oiRedirect{textbook-flickr_ryan_claussen_iris_picture}{http://flic.kr/p/6QTcuX}) 
\oiRedirect{textbook-CC_BY_SA_2}{CC~BY-SA~2.0~license}}
\vfill \ 
\end{minipage}
}{}

% 8

\eoce{\qt{Smoking habits of UK residents\label{smoking_habits_UK_datamatrix}} A survey 
was conducted to study the smoking habits of UK residents. Below is a data 
matrix displaying a portion of the data collected in this survey. Note that 
``$\pounds$" stands for British Pounds Sterling, ``cig" stands for cigarettes, 
and ``N/A'' refers to a missing component of the data. \footfullcite{data:smoking}
\begin{center}
\scriptsize{
\begin{tabular}{rccccccc}
\hline
	& sex 	 & age 	& marital 	& grossIncome 					     & smoke & amtWeekends	& amtWeekdays \\ 
\hline
1 	& Female & 42 	& Single 	& Under $\pounds$2,600 			     & Yes 	 & 12 cig/day   & 12 cig/day \\ 
2 	& Male	 & 44	& Single 	& $\pounds$10,400 to $\pounds$15,600 & No	 & N/A 			& N/A \\ 
3 	& Male 	 & 53 	& Married   & Above $\pounds$36,400 		     & Yes 	 & 6 cig/day 	& 6 cig/day \\ 
\vdots & \vdots & \vdots & \vdots & \vdots 				             & \vdots & \vdots 	    & \vdots \\ 
1691 & Male  & 40   & Single 	& $\pounds$2,600 to $\pounds$5,200   & Yes 	 & 8 cig/day 	& 8 cig/day \\   
\hline
\end{tabular}
}
\end{center}
\begin{parts}
\item What does each row of the data matrix represent?
\item How many participants were included in the survey?
\item Indicate whether each variable in the study is numerical or categorical. If numerical, identify as 
continuous or discrete. If categorical, indicate if the variable is ordinal.
\end{parts}
}{}

\noindent
\textit{JV: Insert exercise showing a data matrix from a bio/genetics study.}

\noindent
\textit{JV: Add questions involving identification of explanatory and response variables.}

%_______________
\subsection{Overview of data collection principles}

% 9

\eoce{\qt{Air pollution and birth outcomes, scope of inference\label{scope_airpoll}} 
Exercise~\ref{study_components_airpoll} introduces a study where researchers 
collected data to examine the relationship between air pollutants and preterm 
births in Southern California. During the study air pollution levels were 
measured by air quality monitoring stations. Length of gestation data were 
collected on 143,196 births between the years 1989 and 1993, and air pollution 
exposure during gestation was calculated for each birth.
\begin{parts}
\item Identify the population of interest and the sample in this study.
\item Comment on whether or not the results of the study can be generalized to the 
population, and if the findings of the study can be used to establish causal relationships.
\end{parts}
}{}

\noindent
\textit{JV: Placeholder for exercise referring back to previous section, replacement for cheaters exercise.}

\begin{comment}
% 10

\eoce{\qt{Cheaters, scope of inference\label{scope_cheaters}} 
Exercise~\ref{study_components_cheaters} introduces a study where researchers 
studying the relationship between honesty, age, and self-control conducted an 
experiment on 160 children between the ages of 5 and 15. The researchers asked 
each child to toss a fair coin in private and to record the outcome (white or black) 
on a paper sheet, and said they would only reward children who report white. 
Half the students were explicitly told not to cheat and the others were not given 
any explicit instructions. Differences were observed in the cheating rates in the
instruction and no instruction groups, as well as some differences across 
children's characteristics within each group.
\begin{parts}
\item Identify the population of interest and the sample in this study.
\item Comment on whether or not the results of the study can be generalized to the 
population, and if the findings of the study can be used to establish causal 
relationships.
\end{parts}
}{}

\end{comment}

% 11

\eoce{\qt{Buteyko method, scope of inference\label{scope_buteyko}} 
Exercise~\ref{study_components_buteyko} introduces a study on using the Buteyko 
shallow breathing technique to reduce asthma symptoms and improve quality of life.
As part of this study 600 asthma patients aged 18-69 who relied on medication for 
asthma treatment were recruited and randomly assigned to two groups: one practiced 
the Buteyko method and the other did not. Those in the Buteyko group experienced,
on average, a significant reduction in asthma symptoms and an improvement in quality 
of life.
\begin{parts}
\item Identify the population of interest and the sample in this study.
\item Comment on whether or not the results of the study can be generalized to the 
population, and if the findings of the study can be used to establish causal 
relationships.
\end{parts}
}{}

\noindent
\textit{JV: Replacement for socioeconomic class + unethical behavior exercise.}

\begin{comment}
% 12

\eoce{\qt{Stealers, scope of inference\label{scope_stealers}} 
Exercise~\ref{study_components_stealers} introduces a study on the relationship 
between socio-economic class and unethical behavior. As part of this study 129 
University of California Berkeley undergraduates were asked to identify themselves 
as having low or high social-class by comparing themselves to others with the most 
(least) money, most (least) education, and most (least) respected jobs. They were 
also presented  with a jar of individually wrapped candies and informed that the
candies were for children in a nearby laboratory, but that they could take some if 
they wanted. After completing some unrelated tasks, participants reported the 
number of candies they had taken. It was found that those who were identified as 
upper-class took more candy than others.
\begin{parts}
\item Identify the population of interest and the sample in this study.
\item Comment on whether or not the results of the study can be generalized to the 
population, and if the findings of the study can be used to establish causal 
relationships.
\end{parts}
}{}

\end{comment}

%_______________
%\subsection{Observational studies and sampling strategies}

\noindent
\textit{JV: Can be used if reworded so that scatterplot graphic is not needed, mostly for the confounding variable question.}

% 19

\eoce{\qt{Internet use and life expectancy\label{internet_life_expactancy}} The 
following scatterplot was created as part of a study evaluating the 
relationship between estimated life expectancy at birth (as of 2014) and 
percentage of internet users (as of 2009) in 208 countries for which such 
data were available.\footfullcite{data:ciaFactbook}

\noindent\begin{minipage}[c]{0.44\textwidth}
\begin{parts}
\item Describe the relationship between life expectancy and percentage of 
internet users.
\item What type of study is this?
\item State a possible confounding variable that might explain this relationship 
and describe its potential effect.
\end{parts} \vspace{15mm}
\end{minipage}
\begin{minipage}[r]{0.55\textwidth}
\hfill\includegraphics[width = 0.87\textwidth]{ch_intro_to_data_oi_biostat/figures/eoce/internet_life_expactancy/internet_life_expactancy.pdf}
\end{minipage}
}{}

% 20

\eoce{\qt{Stressed out, Part I\label{stressed_out_observational}} A study that 
surveyed a random sample of otherwise healthy high school students found that 
they are more likely to get muscle cramps when they are stressed. The study 
also noted that students drink more coffee and sleep less when they are 
stressed.
\begin{parts}
\item What type of study is this?
\item Can this study be used to conclude a causal relationship between 
increased stress and muscle cramps?
\item State possible confounding variables that might explain the observed 
relationship between increased stress and muscle cramps. 
\end{parts}
}{}

\noindent
\textit{JV: Modify to have a different context?}

% 21

\eoce{\qt{Evaluate sampling methods\label{evaluate_sampling_methods}}
A university wants to determine what fraction of its undergraduate 
student body support a new \$25 annual fee to improve the student 
union. For each proposed method below, indicate whether the method 
is reasonable or not.
\begin{parts}
\item Survey a simple random sample of 500 students.
\item Stratify students by their field of study, then sample 10\% of 
students from each stratum.
\item Cluster students by their ages (e.g. 18 years old in one 
cluster, 19 years old in one cluster, etc.), then randomly sample 
three clusters and survey all students in those clusters.
\end{parts}}{}

\noindent
\textit{JV: Replace with a not-psychology study?}

% 23

\eoce{\qt{Haters gonna hate, study confirms\label{scope_haters}} A study 
published in the \textit{Journal of Personality and Social Psychology} asked a 
group of 200 randomly sampled men and women to evaluate how they felt about 
various subjects, such as camping, health care, architecture, taxidermy, 
crossword puzzles, and Japan in order to measure their dispositional attitude 
towards mostly independent stimuli. Then, they presented the participants with 
information about a new product: a microwave oven. This microwave oven does 
not exist, but the participants didn't know this, and were given three 
positive and three negative fake reviews. People who reacted positively to the 
subjects on the dispositional attitude measurement also tended to react 
positively to the microwave oven, and those who reacted negatively also tended 
to react negatively to it. Researcher concluded that ``some people tend to 
like things, whereas others tend to dislike things, and a more thorough 
understanding of this tendency will lead to a more thorough understanding of 
the psychology of attitudes." \footfullcite{Hepler:2013}
\begin{parts}
\item What are the cases?
\item What is (are) the response variable(s) in this study?
\item What is (are) the explanatory variable(s) in this study?
\item Does the study employ random sampling?
\item Is this an observational study or an experiment? Explain your reasoning.
\item Can we establish a causal link between the explanatory and response 
variables?
\item Can the results of the study be generalized to the population at large?
\end{parts}
}{}

% 24

\eoce{\qt{Family size\label{family_size}} Suppose we want to estimate household 
size, where a ``household" is defined as people living together in the 
same dwelling, and sharing living accommodations. If we select students 
at random at an elementary school and ask them what their family size is, 
will this be a good measure of household size? Or will our average be 
biased? If so, will it overestimate or underestimate the true value?
}{}

\noindent
\textit{JV: I like these.}

% 25

\eoce{\qt{Flawed reasoning\label{flawed_reasoning}} Identify the flaw(s) in reasoning 
in the following scenarios. Explain what the individuals in the study should 
have done differently if they wanted to make such strong conclusions.
\begin{parts}
\item Students at an elementary school are given a questionnaire that they 
are asked to return after their parents have completed it. One of the questions 
asked is, ``Do you find that your work schedule makes it difficult for you to 
spend time with your kids after school?" Of the parents who replied, 85\% said 
``no". Based on these results, the school officials conclude that a great 
majority of the parents have no difficulty spending time with their kids 
after school.
\item A survey is conducted on a simple random sample of 1,000 women who 
recently gave birth, asking them about whether or not they smoked during 
pregnancy. A follow-up survey asking if the children have respiratory problems 
is conducted 3 years later, however, only 567 of these women are reached at the 
same address. The researcher reports that these 567 women are representative 
of all mothers.
\item A orthopedist administers a questionnaire to 30 of his patients who do 
not have any joint problems and finds that 20 of them regularly go running. 
He concludes that running decreases the risk of joint problems.
\end{parts}
}{}

% 26

\eoce{\qt{City council survey\label{city_council_survey}} A city council has requested a 
household survey be conducted in a suburban area of their city. The area is broken 
into many distinct and unique neighborhoods, some including large homes, some with 
only apartments, and others a diverse mixture of housing structures. Identify the 
sampling methods described below, and comment on whether or not you think they 
would be effective in this setting.
\begin{parts}
\item Randomly sample 50 households from the city.
\item Divide the city into neighborhoods, and sample 20 households from each 
neighborhood.
\item Divide the city into neighborhoods, randomly sample 10 neighborhoods, 
and sample all households from those neighborhoods.
\item Divide the city into neighborhoods, randomly sample 10 neighborhoods, 
and then randomly sample 20 households from those neighborhoods.
\item Sample the 200 households closest to the city council offices.
\end{parts}
}{}

\noindent
\textit{JV: Replace w/bio one.}

\begin{comment}

% 27

\eoce{\qt{Sampling strategies\label{sampling_strategies}} A statistics student who is curious about the relationship between the amount of time students spend on social networking sites and their performance at school decides to conduct a survey. Various research strategies for collecting data are described below. In each, name the sampling method proposed and any bias you might expect.
\begin{parts}
\item He randomly samples 40 students from the study's population, gives them the survey, asks them to fill it out and bring it back the next day.
\item He gives out the survey only to his friends, making sure each one of them fills out the survey.
\item He posts a link to an online survey on Facebook and asks his friends to fill out the survey.
\item He randomly samples 5 classes and asks a random sample of students from those classes to fill out the survey.
\end{parts}
}{}

\end{comment}

% 28

\eoce{\qt{Reading the paper\label{reading_paper}} Below are excerpts from two 
articles published in the \emph{NY Times}:
\begin{parts}
\item An article titled \emph{Risks: Smokers Found More Prone to Dementia} 
states the following: \footfullcite{news:smokingDementia}
\begin{adjustwidth}{1em}{1em}
{\footnotesize ``Researchers analyzed data from 23,123 health plan members who 
participated in a voluntary exam and health behavior survey from 1978 to 1985, 
when they were 50-60 years old. 23 years later, about 25\% of the group had 
dementia, including 1,136 with Alzheimer's disease and 416 with vascular 
dementia. After adjusting for other factors, the researchers concluded that 
pack-a-day smokers were 37\% more likely than nonsmokers to develop dementia, 
and the risks went up with increased smoking; 44\% for one to two packs a day; 
and twice the risk for more than two packs."}
\end{adjustwidth}
Based on this study, can we conclude that smoking causes dementia later in 
life? Explain your reasoning.
\item Another article titled \emph{The School Bully Is Sleepy} states the 
following: \footfullcite{news:bullySleep}
\begin{adjustwidth}{1em}{1em}
{\footnotesize ``The University of Michigan study, collected survey data from 
parents on each child's sleep habits and asked both parents and teachers to 
assess behavioral concerns. About a third of the students studied were 
identified by parents or teachers as having problems with disruptive behavior 
or bullying. The researchers found that children who had behavioral issues and 
those who were identified as bullies were twice as likely to have shown 
symptoms of sleep disorders."}
\end{adjustwidth}
A friend of yours who read the article says, ``The study shows that sleep 
disorders lead to bullying in school children." Is this statement justified? 
If not, how best can you describe the conclusion that can be drawn from this 
study?
\end{parts}
}{}

%_______________
%\subsection{Experiments}

\noindent
\textit{JV: Merge this one with Part I.}

% 30

\eoce{\qt{Stressed out, Part II\label{stressed_out_experiment}} In a study evaluating the 
relationship between stress and muscle cramps half the subjects are randomly assigned to 
be exposed to increased stressed by being placed into an elevator that falls rapidly and 
stops abruptly and the other half are left at no or baseline stress.
\begin{parts}
\item What type of study is this?
\item Can this study be used to conclude a causal relationship between increased stress 
and muscle cramps?
\end{parts}
}{}


% 32

\eoce{\qt{Vitamin supplements\label{vitamin_supplement}} In order to assess the effectiveness 
of taking large doses of vitamin C in reducing the duration of the common cold, 
researchers recruited 400 healthy volunteers from staff and students at a university. A 
quarter of the patients were assigned a placebo, and the rest were evenly divided 
between 1g Vitamin C,  3g Vitamin C, or 3g Vitamin C plus additives to be taken at onset 
of a cold for the following two days. All tablets had identical appearance and packaging.
The nurses who handed the prescribed pills to the patients knew which patient received 
which treatment, but the researchers assessing the patients when they were sick did not. 
No significant differences were observed in any measure of cold duration or severity 
between the four medication groups, and the placebo group had the shortest duration of 
symptoms.\footfullcite{Audera:2001}
\begin{parts}
\item Was this an experiment or an observational study? Why?
\item What are the explanatory and response variables in this study?
\item Were the patients blinded to their treatment?
\item Was this study double-blind?
\item Participants are ultimately able to choose whether or not to use the pills 
prescribed to them. We might expect that not all of them will adhere and take their 
pills. Does this introduce a confounding variable to the study? Explain your reasoning.
\end{parts}
}{}

\noindent
\textit{JV: Include more realistic open-ended study design questions? Or eliminate entirely?}

\begin{comment}
% 34

\eoce{\qt{Music and learning\label{music_learning}} You would like to conduct an experiment in 
class to see if students learn better if they study without any music, with music that 
has no lyrics (instrumental), or with music that has lyrics. Briefly outline a design for 
this study.
}{}

% 35

\eoce{\qt{Soda preference\label{soda_preference}} You would like to conduct an experiment in 
class to see if your classmates prefer the taste of regular Coke or Diet Coke. Briefly 
outline a design for this study.
}{}

\end{comment}

% 36

\eoce{\qt{Exercise and mental health\label{exercise_mental_health}} A researcher is interested 
in the effects of exercise on mental health and he proposes the following study: Use 
stratified random sampling to ensure representative proportions of 18-30, 31-40 and 41-
55 year olds from the population. Next, randomly assign half the subjects from each age 
group to exercise twice a week, and instruct the rest not to exercise. Conduct a mental 
health exam at the beginning and at the end of the study, and compare the results.
\begin{parts}
\item What type of study is this? 
\item What are the treatment and control groups in this study?
\item Does this study make use of blocking? If so, what is the blocking variable?
\item Does this study make use of blinding?
\item Comment on whether or not the results of the study can be used to establish a 
causal relationship between exercise and mental health, and indicate whether or not the 
conclusions can be generalized to the population at large.
\item Suppose you are given the task of determining if this proposed study should get 
funding. Would you have any reservations about the study proposal?
\end{parts}
}{}

\noindent
\textit{JV: Replace with biology question in similar style.}

\begin{comment}
% 37

\eoce{\qt{Chia seeds and weight loss\label{chia_weight_lostt}} Chia Pets -- those terra-cotta 
figurines that sprout fuzzy green hair -- made the chia plant a household name. But chia 
has gained an entirely new reputation as a diet supplement.  In one 2009 study, a team 
of researchers recruited 38 men and divided them randomly into two groups: treatment or 
control. They also recruited 38 women, and they randomly placed half of these 
participants into the treatment group and the other half into the control group. One 
group was given 25 grams of chia seeds twice a day, and the other was given a placebo. 
The subjects volunteered to be a part of the study. After 12 weeks, the scientists found 
no significant difference between the groups in appetite or weight loss. 
\footfullcite{Nieman:2009}
\begin{parts}
\item What type of study is this? 
\item What are the experimental and control treatments in this study?
\item Has blocking been used in this study? If so, what is the blocking variable?
\item Has blinding been used in this study?
\item Comment on whether or not we can make a causal statement, and indicate whether or 
not we can generalize the conclusion to the population at large.
\end{parts}
}{}

\end{comment}

%_______________
\subsection{Examining numerical data}

% 38

\eoce{\qt{Mammal life spans\label{mammal_life_spans}} Data were collected on life spans (in 
years) and gestation lengths (in days) for 62 mammals. A scatterplot of life span versus 
length of gestation is shown below. \footfullcite{Allison+Cicchetti:1975}

\noindent\begin{minipage}[c]{0.44\textwidth}
\begin{parts}
\item What type of an association is apparent between life span and length of gestation?
\item What type of an association would you expect to see if the axes of the plot were reversed, i.e. if we plotted length of gestation versus life span?
\item Are life span and length of gestation independent? Explain your reasoning.
\end{parts}\vspace{6mm}
\end{minipage}
\begin{minipage}[c]{0.55\textwidth}
\begin{center}
\includegraphics[width = 0.86\textwidth]{ch_intro_to_data_oi_biostat/figures/eoce/mammal_life_spans/mammal_life_spans_scatterplot.pdf}
\end{center}
\end{minipage}
}{}

% 39

\eoce{\qt{Associations\label{association_plots}} Indicate which of the plots show a \\[1mm]
\noindent\begin{minipage}[b]{0.35\textwidth}
\begin{parts}
\item positive association
\item negative association
\item no association
\end{parts}
Also determine if the positive and negative associations are linear or nonlinear. Each 
part may refer to more than one plot. \vspace{30mm}
\end{minipage}
\begin{minipage}[b]{0.62\textwidth}
\hfill\includegraphics[width = 0.95\textwidth]{ch_intro_to_data_oi_biostat/figures/eoce/association_plots/association_plots.pdf}
\end{minipage}
}{}

% 41

\eoce{\qt{Reproducing bacteria\label{reproducing_bacteria}} Suppose that there is only 
sufficient space and nutrients to support one million bacterial cells in a petri dish. 
You place a few bacterial cells in this petri dish, allow them to reproduce freely, and 
record the number of bacterial cells in the dish over time. Sketch a plot representing 
the relationship between number of bacterial cells and time.
}{}

\noindent
\textit{JV: Re-write for different contexts -- identifying sample vs. population means}

% 42

\eoce{\qt{Sleeping in college\label{college_sleeping}} A recent article in a college newspaper 
stated that college students get an average of 5.5 hrs of sleep each night. A student who 
was skeptical about this value decided to conduct a survey by randomly sampling 25 
students. On average, the sampled students slept 6.25 hours per night. Identify which 
value represents the sample mean and which value represents the claimed population mean.
}{}

% 43

\eoce{\qt{Parameters and statistics\label{parameters_stats}} Identify which value represents 
the sample mean and which value represents the claimed population mean.
\begin{parts}
\item American households spent an average of about \$52 in 2007 on Halloween 
merchandise such as costumes, decorations and candy. To see if this number had changed, 
researchers conducted a new survey in 2008 before industry numbers were reported. The 
survey included 1,500 households and found that average Halloween spending was \$58 per 
household.
\item The average GPA of students in 2001 at a private university was 3.37. A survey 
on a sample of 203 students from this university yielded an average GPA of 3.59 in 
Spring semester of 2012.
\end{parts}
}{}

% 46

\eoce{\qt{Medians and IQRs} For each part, compare distributions (1) and (2) based on their medians and IQRs. You do not need to calculate these statistics; simply state how the medians and IQRs compare. Make sure to explain your reasoning. 
\begin{multicols}{2}
\begin{parts}
\item (1) 3, 5, 6, 7, 9 \\
(2) 3, 5, 6, 7, 20
\item (1) 3, 5, 6, 7, 9 \\
(2) 3, 5, 8, 7, 9
\item (1) 1, 2, 3, 4, 5 \\
(2) 6, 7, 8, 9, 10
\item (1) 0, 10, 50, 60, 100 \\
(2) 0, 100, 500, 600, 1000
\end{parts}
\end{multicols}
}{}

% 47

\eoce{\qt{Means and SDs} For each part, compare distributions (1) and (2) based on their means and standard deviations. You do not need to calculate these statistics; simply state how the means and the standard deviations compare. Make sure to explain your reasoning. \textit{Hint:} It may be useful to sketch dot plots of the distributions.
\begin{multicols}{2}
\begin{parts}
\item (1) 3, 5, 5, 5, 8, 11, 11, 11, 13 \\
(2) 3, 5, 5, 5, 8, 11, 11, 11, 20 \\
\item (1) -20, 0, 0, 0, 15, 25, 30, 30 \\
(2) -40, 0, 0, 0, 15, 25, 30, 30
\item (1) 0, 2, 4, 6, 8, 10 \\
(2) 20, 22, 24, 26, 28, 30
\item (1) 100, 200, 300, 400, 500 \\
(2) 0, 50, 300, 550, 600
\end{parts}
\end{multicols}
}{}

% 50

\eoce{\qt{Mix-and-match} Describe the distribution in the histograms below and match them to the box plots. \\
\begin{center}
\includegraphics[width=\textwidth]{ch_intro_to_data_oi_biostat/figures/eoce/hist_box_match/hist_box_match.pdf}
\end{center}
}{}

% 51

\eoce{\qt{Air quality\label{air_quality_durham}} Daily air quality is measured by the air 
quality index (AQI) reported by the Environmental Protection Agency. This index reports 
the pollution level and what associated health effects might be a concern. The index is 
calculated for five major air pollutants regulated by the Clean Air Act and takes values 
from 0 to 300, where a higher value indicates lower air quality. AQI was reported for a 
sample of 91 days in 2011 in Durham, NC. The relative frequency histogram below shows 
the distribution of the AQI values on these days. \footfullcite{data:durhamAQI:2011}
\begin{minipage}[c]{0.55\textwidth}
\begin{parts}
\item Estimate the median AQI value of this sample.
\item Would you expect the mean AQI value of this sample to be higher or lower than the 
median? Explain your reasoning.
\item Estimate Q1, Q3, and IQR for the distribution.
\item Would any of the days in this sample be considered to have an unusually low or 
high AQI? Explain your reasoning.
\end{parts}
\end{minipage}
\begin{minipage}[c]{0.45\textwidth}
\begin{center}
\includegraphics[width = \textwidth]{ch_intro_to_data_oi_biostat/figures/eoce/air_quality_durham/air_quality_durham_rel_freq_hist.pdf} 
\end{center}
\end{minipage}
}{}

\noindent
\textit{JV: Do histogram vs. boxplot exercises for real data, incl. biological example.}

\begin{comment}

% 53

\eoce{\qt{Histograms vs. box plots\label{hist_vs_box}} Compare the two plots below. What 
characteristics of the distribution are apparent in the histogram and not in the box 
plot? What characteristics are apparent in the box plot but not in the histogram?
\begin{center}
\includegraphics[width = 0.6\textwidth]{ch_intro_to_data/figures/eoce/hist_vs_box/hist_vs_box.pdf}
\end{center}
}{}

% 54

\eoce{\qt{Marathon winners\label{marathon_winners}} The histogram and box plots below show the distribution of finishing times for male and female winners of the New York Marathon between 1970 and 1999.
\begin{center}
\includegraphics[width=0.6\textwidth]{ch_intro_to_data/figures/eoce/marathon_winners/marathon_winners_hist_box.pdf}
\end{center}
\begin{parts}
\item What features of the distribution are apparent in the histogram and not the box plot? What features are apparent in the box plot but not in the histogram?
\item What may be the reason for the bimodal distribution? Explain.
\item Compare the distribution of marathon times for men and women based on the box plot shown below.
\begin{center}
\includegraphics[width=0.6\textwidth]{ch_intro_to_data/figures/eoce/marathon_winners/marathon_winners_gender_box.pdf}
\end{center}
\item The time series plot shown below is another way to look at these data. Describe what is visible in this plot but not in the others.
\end{parts}
\begin{center}
\includegraphics[width=0.75\textwidth]{ch_intro_to_data/figures/eoce/marathon_winners/marathon_winners_time_series.pdf} \\
\end{center}
}{}

\end{comment}

\noindent
\textit{JV: Adapt for more realistic examples of skew.}

\begin{comment}
% 56

\eoce{\qt{Distributions and appropriate statistics, Part II
\label{dist_shape_housing_alcohol_salary}} For each of the following, state 
whether you expect the distribution to be symmetric, right skewed, or left skewed. 
Also specify whether the mean or median would best represent a typical observation 
in the data, and whether the variability of observations would be best represented 
using the standard deviation or IQR. Explain your reasoning.
\begin{parts}
\item Housing prices in a country where 25\% of the houses cost below \$350,000, 
50\% of the houses cost below \$450,000, 75\% of the houses cost below \$1,000,000 
and there are a meaningful number of houses that cost more than \$6,000,000.
\item Housing prices in a country where 25\% of the houses cost below \$300,000, 
50\% of the houses cost below \$600,000, 75\% of the houses cost below \$900,000 
and very few houses that cost more than \$1,200,000.
\item Number of alcoholic drinks consumed by college students in a given week. 
Assume that most of these students don't drink since they are under 21 years old, 
and only a few drink excessively.
\item Annual salaries of the employees at a Fortune 500 company where only a few 
high level executives earn much higher salaries than the all other employees.
\end{parts}
}{}

\end{comment}

\noindent
\textit{JV: Nice setup for robustness exercise. Adapt context?}

% 61

\eoce{\qt{Income at the coffee shop\label{income_coffee_shop}} The first histogram 
below shows the distribution of the yearly incomes of 40 patrons at a college 
coffee shop. Suppose two new people walk into the coffee shop: one making 
\$225,000 and the other \$250,000. The second histogram shows the new income 
distribution. Summary statistics are also provided. \\
\begin{minipage}[c]{0.57\textwidth}
\includegraphics[width=\textwidth]{ch_intro_to_data_oi_biostat/figures/eoce/income_coffee_shop/income_coffee_shop.pdf}
\end{minipage}
\begin{minipage}[c]{0.4\textwidth}
\begin{center}
\begin{tabular}{rrr}
\hline
        & (1)       & (2) \\ 
\hline
n       & 40        & 42 \\ 
Min.    & 60,680    & 60,680 \\ 
1st Qu. & 63,620    & 63,710 \\ 
Median  & 65,240    & 65,350 \\ 
Mean    & 65,090    & 73,300 \\ 
3rd Qu. & 66,160    & 66,540 \\ 
Max.    & 69,890    & 250,000 \\ 
SD      & 2,122     & 37,321 \\ 
\hline
\end{tabular}
\end{center}
\end{minipage}
\begin{parts}
\item Would the mean or the median best represent what we might think of as a 
typical income for the 42 patrons at this coffee shop? What does this say about 
the robustness of the two measures?
\item Would the standard deviation or the IQR best represent the amount of 
variability in the incomes of the 42 patrons at this coffee shop? What does 
this say about the robustness of the two measures?
\end{parts}
}{}

% 62

\eoce{\qt{Midrange\label{midrange}} The \textit{midrange} of a distribution is defined as 
the average of the maximum and the minimum of that distribution. Is this statistic 
robust to outliers and extreme skew? Explain your reasoning
}{}

\noindent
\textit{JV: Need a data transformation exercise or two.}

%_______________
\subsection{Considering categorical data}

\noindent
\textit{JV: Use different data.}

% 66

\eoce{\qt{Views on immigration\label{immigration}} 910 randomly sampled registered 
voters from Tampa, FL were asked if they thought workers who have illegally 
entered the US should be (i) allowed to keep their jobs and apply for 
US citizenship, (ii) allowed to keep their jobs as temporary guest workers 
but not allowed to apply for US citizenship, or (iii) lose their jobs and 
have to leave the country. The results of the survey by political ideology 
are shown below.\footfullcite{survey:immigFL:2012}
\begin{center}
\begin{tabular}{l l c c c c}
                        &                           & \multicolumn{3}{c}{\textit{Political ideology}} \\
\cline{3-5}
                        &                           & Conservative  & Moderate  & Liberal   & Total \\
\cline{2-6}
                        & (i) Apply for citizenship & 57            & 120       & 101       & 278 \\
                        & (ii) Guest worker         & 121           & 113       & 28        & 262 \\
\raisebox{1.5ex}[0pt]{\emph{Response}} & (iii) Leave the country    & 179       & 126       & 45        & 350 \\ 
                        & (iv) Not sure             & 15            & 4         & 1         & 20\\
\cline{2-6}
                        & Total                     & 372           & 363       & 175       & 910
\end{tabular}
\end{center}
\begin{parts}
\item What percent of these Tampa, FL voters identify themselves as conservatives?
\item What percent of these Tampa, FL voters are in favor of the citizenship option?
\item What percent of these Tampa, FL voters identify themselves as conservatives 
and are in favor of the citizenship option?
\item What percent of these Tampa, FL voters who identify themselves as 
conservatives are also in favor of the citizenship option? What percent of 
moderates share this view? What percent of liberals share this view?
\item Do political ideology and views on immigration appear to be independent? 
Explain your reasoning.
\end{parts}
}{}

\noindent
\textit{JV: Another contingency table exercise, as well as something with numerical data across groups.}